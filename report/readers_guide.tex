% -*- mode: latex; mode: auto-fill; coding: utf-8; -*-

\chapter*{Readers Guide}
This thesis consists of three parts. Part I presents analytical
theories. This part includes statics, dynamics, continuum, and fracture
mechanics. Futhermore, elasticity theory and basic linear algebra is
introduced. Part II presents discrete theories. Here the fundamental
equilibrium framework and the finite element method are explained in
detail and used to construct a simulation model. Part III explains how
the simulation software is designed and implemented. Parallel
execution is introduced and the simulation software constructed is
evaluated.
%
The following is a brief summary of each chapter: \\ 

Chapter \ref{chapter:introduction} introduces the reader to the field
of simulated surgery and the potential of computer aided simulations
in general. Here the motivation behind this thesis is stated.
%
Chapter \ref{chapter:problem_domain} explains the surgical procedure
of wisdom tooth removal step by step. The physical phenomena of
interest are identified and the concepts real-time and simulator are
defined. The main objectives in the thesis is stated and related work
are presented.

\section*{Part I - Analytic Theories}
Chapter \ref{chapter:physics} introduces the fundamental physics
used. Work, energy, and equilibrium are defined before moving on to
continuum mechanics where the concepts deformation, elasticity,
stress and strain are introduced.
%
Chapter \ref{chapter:mathematics} explains the basic linear algebra
used.

\section*{Part II - Discrete Theories}
Chapter \ref{chapter:equilibrium_framwork} introduces the fundamental
structure for equilibrium problems in general.
%
Chapter \ref{chapter:finite_element_method} describes the finite
element method. The discretization of the continuum body and how
to assemble the system equations are explained in detail.
%
Chapter \ref{chapter:applying_finite_element_method} explains how to
apply the finite element method and a concrete solver technique is
presented.

\section*{Part III - Implementation}
Chapter \ref{chapter:simulation_model} presents the simulation
software developed.
%
Chapter \ref{chapter:parallel_execution} introduces the concept of
parallel execution and explains how we utilize NVIDIA's CUDA
technology.
%
Chapter \ref{chapter:helper_tools} explains the developed tools for
interacting with the simulation software and how to visualise
tensors.
%
Chapter \ref{chapter:results} presents the results obtained.
%
Chapter \ref{chapter:future_work} is a discussion on possible
improvements.
%
Chapter \ref{chapter:conclusion} summarizes and concludes on the
results obtained. 
