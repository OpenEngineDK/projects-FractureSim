% -*- mode: latex; mode: auto-fill; coding: utf-8; -*-

During the development of the simulation prototype 
%While working intensively with this thesis 
lots of improvements came
to mind. When considering the endless interdependent
components of the final prototype, ideas, questions or improvements
arise to each of the individual components. It seems like whenever we look
into how things are connected and how changes would propagate though the
system new challenges arise.
%
In this section we have limited the discussion of any future work by only
considering improvements of the prototype that will lead towards a
simulator that can be evaluated by dentists. To form a decent
evaluation basis for a dentist, improvements must be made.
It would be very helpful to get evaluation feedback from a dentist
especially on the sense of touch when the fragmentation is
performed. Analysing the fragments of the tooth by comparing the
simulated fragmentation to actual tooth fragments could also be very
interesting. \\

% Shape function
\subsubsection*{Improving the precision on the pre-calculations}
The pre-calculation of the shape function derivatives is based on the
geometry of the elements. Currently the global position of the
elements can cause significant floating point inaccuracy leading to
inaccurate shape function derivatives. To minimize the error the shape
function derivatives could be pre-calculated in local space
coordinates instead of global. 
%
The geometry of the
elements are represented by floating point numbers which have a
relative representation error \citebook{page~9}{notes:oleby}. So the
error increases with the distance from origo of the global coordinate
system. By translating the geometry of
the elements into origo and scaling it to unity the error can be
minimized. \\


% Apply forces
\subsubsection*{Improving how to apply forces}
An essential part of the simulator is the interaction. To simulate a
reaction something has to act on the system changing the state of
equilibrium. As described in section \vref{sec:modifiers} all
interaction is carried out through the 
modifiers. The Projective Displacement modifier is the modifier used for
simulating interactive tools like the elevator from the test
scenarios. As implied by its name, the modifier interacts with the
system through nodal displacements. So when there is detectable
collision between tool and object the colliding surface nodes of the
object are displaced until they are free of collision. At the point of
contact between tool and object all forces are applied through
deformation of the object's surface. Using brittle
materials even the slightest deformation causes a huge amount of
internal stress, which makes the interaction between tool and object
difficult. When the tool touches the surface of the object in a way
that seems very gently on the screen, the small surface deformation is
actually causing the object to fracture immediately. This would
probably be the correct response if we did hit the object with a sharp
tool using great force, but this is not the case here. Basically the
problem is the method used when applying the forces as
displacement. By displacing the surface elements, at the
point of contact the internal forces peak in a single iteration before
they get the chance to propagate to neighbouring elements over
multiple iterations. This has to do with the explicit approach
used. Since each element is solved explicit they only interact when
nodal force contributions are added at the end of each
iteration. Therefore forces needs to propagate through the elements
over multiple iterations.
%
A more gentle method for applying forces could improve the interaction
between tool and object. The displacement could be gradually applied
to the surface elements or distributed over multiple elements within a
certain distance to compensate for the propagation.

% Improve strategy for measuring stress
\subsubsection*{Improving the method for measuring internal stress}
In relation to how forces are applied into the system the method for
determining the maximum level of internal stress needs
improvement. The current method determines the maximal level of stress
by considering each individual element separately. 
%
If the internal stress peaks in a single element
%due to a small displacement
the fracturing limit is exceeded upon impact
causing the object to start fracturing before the stress has
propagated properly through the element mesh. Instead the maximum
level of stress could be measured by interpolating the stresses from
neighbouring elements hereby reducing peak values. 

% Crack tracking
\subsubsection*{Improving the crack tracking algorithm}
In general we have obtained good results from the local crack
tracking algorithm implemented. But there are improvements to be
made. If the crack initialization has been initiated due to an
internal stress peak in a single element the algorithm performs
poorly.
%
It performs poorly because the internal stress has not propagated
properly from the point of impact and throughout the element mesh.
As a result some elements have a magnitude of the principal stress values is
alternating around zero. This causes the maximum principal stress
direction to alter between the three principal directions in a more or
less random manner.
%
The crack tracking algorithm relies on the principal stress directions
so as a consequence of the randomness the failure surface
determination can 'spin out of control'. Since every element only can
crack once and the number of elements is finite, the algorithm is
deterministic at all times. But the crack propagation can in rare
cases form a spiral like a winding staircase. This only occurs when
the crack is propagating through parts of the body with internal
principal stress values around zero and could be avoided by improving
the stress measure as mentioned earlier or by improving the crack
tracking algorithm to perform better with seemingly random stress
directions.

% Re-meshing
\subsubsection*{Re-meshing of the fractured body}
When the crack tracking algorithm has determined a failure surface the
intersecting elements should be separated accordingly. The prototype
does not handle the actual separation of the fragments. To facilitate a
proper evaluation of the size and shape of the fragments, separation
has to be performed. How each element has been cracked is registered
as crack points determining the intersection between the crack plane
and the element edges. This representation is used for visualizing
the crack surface and can be used to separate the element mesh.

% Composite materials
\subsubsection*{Support composite materials}
A wisdom tooth consists of different materials, primarily dentin.
The material properties of dentin are not the same throughout a tooth,
different material subgroups like superficial, middle,
and deep dentin have different material properties.
%
Besides dentin a layer of enamel is covering the crown, this material
is also divided into subgroups \citebook{page~324}{Giannini2004322}.
%
Currently the prototype only supports homogeneous materials which
means that all elements have the same material properties. By
implementing support for different material properties in each element,
inhomogeneous and composite materials could be simulated. It would
require extending the mesh file format, the internal data structures
and making small changes to the finite element solving technique. It
could be very interesting to evaluate and test how the crack tracking
performs in composite materials.


% Registering tooth
\subsubsection*{Improve tooth mesh}
The tooth mesh used in the test scenarios is modelled by hand roughly
shaped to form the real tooth borrowed. Since only the surface of the
real tooth has been taken into consideration the model does not
include any internal cavities. Instead a model could be created from
a volumetric x-ray image by registering the surface curvature in each
layer and assemble these to form a volumetric contour of the tooth which
includes cavities. The internal cavities will effect the propagation
of internal stress and strain making the tooth simulation more
compliant with the actual scenario.


% GUI
\subsubsection*{Improving the graphical user interface}
The prototype suffers from a lot of hard coded properties and
setups. Few parameters, like which mesh and material should be used,
can be passed as command-line arguments at start-up. Other than that, the
code needs to be changed and recompiled. A graphical user interface capable
of changing the simulated scenario at run-time is necessary if
non-technical personal should evaluate the simulator.


% Haptic
\subsubsection*{Support for haptic devices}
To facilitate a proper evaluation of the simulated fragmentation
process the input devices must be improved. Feedback from dentists
on how the tools manoeuvre in space, how the twisting action of the
elevator tool complies to the actual scenario etc. would be helpful.
%
Using a standard keyboard and mouse is not sufficient if we are
interested in evaluating the simulated response of user interactions.
When using the simulated tool the user tends to lose
the sense of how much force he or she actually applies to the
object. Especially with brittle materials where even a very small
deformation causes huge amounts of internal stress. 
%
Implementing support for a haptic device with force feedback would be
necessary for the dentists to evaluate the sense of touch and
manoeuvrability in the simulator. 

% \subsubsection*{Hardware }
% With the launch of Snow Leopard, Apple, NVIDIA, and the Khronos Group
% have provided a hardware independent alternative to CUDA (to get
% hardware independance).
% }

% \section{Optimizing the geometry for visualization}
% As a pre-computational step the vertex-pool could be sorted, so all
% surface vertices are located at the start of the vertex-pool. When all
% surface vertices are located together, we can render the surface
% faster.
