% -*- mode: latex; mode: auto-fill; coding: utf-8; -*-

%\chapter{Introduction}
% - simulator
% - real time
% - soft tissue + hård materiale
% Set the scene, fixed object, apply forces, eventually break the
% object into smaller pieces.

The research field of computer simulated surgery has been a
topic of increased interest in recent years.
%
The current research has reached a level that facilitates
real-time simulation of deformations in soft organic materials. 
%in high-resolution on consumer hardware. 
Using computer aided simulations as an educational
tool or an interactive training facility has a lot of potential when
it comes to imparting new knowledge to students or
trainees. Surgical procedures can be studied and practiced iteratively
through extended feedback including any imaginable visual
information. While performing the simulated surgery instructions and
guidance could be presented, any potential risk could be pointed out,
and unexpected critical scenarios could occur. 
Simulated surgery has the potential to enrich the
learning process on so many levels.  
%
From the perspective of
the student the main objective of using simulated surgery is to
acquire certain, often very difficult, skills in a risk-free
environment. Introducing surgical certificates based
on simulations is an option within the near future. Professional approval
based on simulated scenarios has been used for many years when
educating and certifying pilots. \\

% Challenges and limitations 
The field of simulated surgery still faces great
challenges especially when it comes to the sense of touch. It is
very difficult to simulate the correct feedback on the tool being
used e.g. the weight, friction, momentum, manoeuvrability
etc. Simulation models based on the laws of physics will
always be idealized and respond as predicted by theory. Sometimes
theory and practice is not consistent. \\
%
% With attention focused on improving the visual feedback there
% is a lot to benefit from simulated surgeries. 

%
The development within computer aided visualization has been
tremendous during the last decades. The research within this field
has benefited heavily from the huge investments made by the
entertainment industry constantly striving towards improving the level
of realism. 
%
Simulating organic material like human tissue or organs require a
model capable of representing the complex structures and
the theoretical laws of behaviour. 
%
In the field of applied mathematics a generalized method for
conducting numerical modeling of physical systems was first well
defined in the late 1960's and early 1970's. 
%
Although these mathematical models are capable of capturing
complicated problems they tend to get very complex and computationally
expensive.
%
Within recent years the introduction of multi-core hardware and the
development in GPU programming languages, suddenly has made these
models, based on the actual laws of physics, realisable. By taking
advantage of the parallel architecture of the GPU, the computational
resources have increased by orders of magnitude assuming the problem
can be solved in a parallel environment. \\

%In recent years the research field within computer simulated surgeries
%has increased drastically. 
%
% The increasing interest for simulated surgeries used as an educational
% tool or interactive training facility is probably due to the
% fact that the current level of research now facilitates
% real-time simulation of organic materials in high-resolution.
% %
% The mathematical models capable of representing problems within the
% problem domain of simulating complex structures have been developed
% in modern times. In applied mathematics a generalized method for
% conducting numerical modeling of physical systems was first well
% defined in the late 1960's and early 1970's. 
% %
% The development within computer aided visualizations has been
% tremendous during the last decades. The research within this field
% have benefited heavily from the huge investments made from the
% entertainment industry constantly trying to improve the level of
% realism. 
% %
% Simulating organic material like human tissue and organs requires huge
% computational efforts, but with the introduction of multi-core
% hardware and the development in GPU programming languages, models
% based on the laws of physics suddenly becomes realisable. \\
%

% Modern consumer hardware has gradually increased in
% computational resources to a level now facilitating real-time
% execution of interactive 
%

% Modern consumer hardware has gradually reached a level that now 
% facilitates real-time execution of plausible simulations based on
% mathematical models incorporating the laws of physics

% Background history
This thesis is motivated by the current demand for an educational tool
teaching a specific dental procedure. Inspired by related
simulated surgeries The Aarhus School of Dentistry has requested a
simulator for teaching the common procedure of wisdom tooth
extraction. Performing the surgical procedure requires certain skills
and must be
performed with caution to prevent serious nerve damage. All
surgery involves risks, in this case the patient's tongue or part of
the lip could become permanently numb due to nerve damage. This
emphasizes the importance of a proficiency before
operating on patients. Besides lots of lessons and videos teaching how
to perform the operation, students practise on a rubber doll known
as a \defit{phantom}. Operating on a phantom is the last step before the
students are expected to perform supervised operations on
patients. According to Professor, dr. odont. Søren Schou, Aarhus
University, there is a huge educational gap between operating on a
phantom and a real patient. \\

%This thesis deals with the design and implementation of a model
%This thesis deals with the development of a simulation model based on
%structural analysis using the laws of physics to simulate 
This thesis deals with the development of a model for simulating one
specific part of the complete surgical procedure of wisdom tooth
removal.
%
Attention is focused on how to simulate the fragmentation
that will separate the crown of the tooth from its roots. 
By solving this particular problem we get one step closer
towards the development of a complete simulator, handling the entire surgical
procedure, hereby providing an educational tool
with the purpose of minimizing the gap between operating on a phantom
and a real patient. \\

The simulation model
is based on the finite element method which is a widely accepted
method for structural analysis. The physical laws from the field of
continuum and fracture mechanics concerning deformation and
fragmentation of solids are applied to obtain as realistic
results as possible with the computational resources available. \\

% Explain to humans what the simulator is - MVC pattern
Basically the simulator consists of three main parts. A visual,
an interactive, and a computational part. The visual part is
responsible for delivering real-time three-dimensional images on the
screen. The interactive part handles the user interaction and
simulates the dental tools accordingly. The computational part 
is responsible for solving the equations defined by the method used
for the structural analysis. \\

% Interdisciplinary approach
It requires an interdisciplinary approach to develop a framework for
simulating deformations and possible fractures in solid objects. 
From the field of dentistry, knowledge on how to perform the actual surgical
procedure, is needed. 
%
Simulating real world phenomena of
fracturing solid objects involves theoretical physics and mathematics, as well as
engineering disciplines of predicting how solids behave under
stress. 
%
Constructing a computational model representing the complex structures
that incorporates the laws of physics suitable for high performance
parallel execution is a computer science discipline. Implementing the
simulator using the proper technologies, benchmarking the
software model and validating the results are all within the
field of computer science. 

% The main emphasis here is on how to
% assemble the relevant theory in the construction of a valid model what
% will be real-time executable

