% -*- mode: latex; mode: auto-fill; coding: utf-8; -*-

% Intro
The main objective of this thesis was to construct a simulation model
capable of conducting real-time structural analysis of stress and
strain to be used for crack surface prediction. \\

%
For the structural analysis we constructed a simulation model based upon
the finite element method and a solver based on the Total Lagrangian
Explicit Dynamics technique aimed towards parallel execution. 
%
% FEM
% The finite element method is one of the most widely accepted methods for
% structural analysis. 
The finite element method with its pure style is so elegant and
yet so complex it takes years to fully master.
% - Worth the effort
Once the method has been implemented as the fundamental framework for
representing and solving the problem domain its true potential emerges. 
% - Stable, predictable, great results -> based on physics e.g. deformation, stress/strain
According to our experiences the method turns out to be very
stable, predictable and beneficial when analysing the behaviour of stress
and strain. \\

% 
Recall from the problem statement that in the context of simulated
surgery the demand for a robust model, real-time execution, and
plausible results is favoured over accuracy.
%
% Problem is the critical time step
Challenges arise when simulating small objects like a human tooth
made of a brittle material like dentin. 
%
When using an explicit solver we need to respect the boundary
conditions on the size of the time step since this is crucial for the
numerical stability of the method.
%
The critical time step is effectively reduced when increasing the
material stiffness or decreasing the smallest edge 
length in the object simulated.
%
The real-time analysis conducted in section
\vref{sec:realtime_analysis} illustrates that with the given
object size and material we cannot achieve real-time simulation which
strictly complies with the theoretical laws of physics.
% Compromise
If we compromise the
theoretical laws of physics by increasing the material density of the tooth we
are allowed to extrapolate simulation time by a larger time step hereby
facilitating real-time execution. In the sense of physical
interpretation the change in density, and hereby the external forces
required for deformation, is out of proportions. But in the sense of 
simulated response the prediction of the crack surface looks very
promising. The crack surface determined seems unaffected since the
same material stiffness is used as elaborated in 
section \vref{sec:fragmentation_with_varying_density}. \\

%
Benchmarking the simulation model with the fragmentation scenario as
described in section \vref{sec:results_fragmentation_tooth} reveals great
potential. 
%
The system equations representing the tooth model can be solved
approximately $20.000$ times per second simulated on hardware as
described in appendix \appref{sec:test-machine-desktop} without
fine-tuning the CUDA implementation. This easily accommodates
real-time interaction and visual feedback at real-time frame rates ($>
25$ FPS). \\

% Crack Tracking
% - Performs well in general - Fracture Mechanics
We have presented a local crack tracking algorithm based upon relevant 
theory from fracture mechanics in particular the principle of maximum
stress direction. 
%
Though improvements can be made as explained in section
\vref{chapter:future_work} we believe the method applied 
shows great potential towards fracturing solid brittle materials.
%
In general the failure surface as predicted by the
crack tracking algorithm looks very promising. The location and
curvature of the failure surface corresponds to the stress analysis
and the intuition of how the object would actually fracture. \\

% - Crack Tracking + Explicit solver = challenges
Using this crack tracking scheme in direct correlation with
the stress and strain analysis conducted did cause a few
shortcomings. As pointed out in section \vref{chapter:future_work} the
approach used for applying forces and 
measuring stress needs improvement to obtain a less sensitive
interaction. 
%
To improve this method any further we need feedback on how actual teeth
fracture in the given surgical scenario. A thorough comparison of simulated
contra actual tooth fragments, requires research beyond the scope of a
single master thesis.



