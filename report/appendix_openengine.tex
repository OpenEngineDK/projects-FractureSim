% -*- mode: latex; mode: auto-fill; -*-

\chapter{Simulator Software}
\section{CD-ROM}
On the enclosed CD-ROM you will find the following:

\begin{itemize}
\item The source code for the entire simulation software.

\item The source code for the OpenEngine framework. 

\item The volumetric meshed as described in appendix
  \appref{chapter:test_data}.

\item Videos and screenshots from the simulations performed.

\item This thesis in pdf-format.
\end{itemize}

\section{OpenEngine Installation}
\label{sec:openengine_installation}
The OpenEngine framework
must be obtained and installed along with a few other tools and
libraries to reproduce the results obtained in this thesis. 
%For install instructions see: \code{http://www.openengine.dk}. 
On the enclosed CD the entire
source code can be found along with the OpenEngine framework and all
resources such as test data and scenario setups.
%
Setting up the different test scenarios requires rebuilding the source
since the implementation is only a prototype. The build system
is based on cmake and tested on Windows Vista, Ubuntu Linux and Mac OS
X. All dependent tools and libraries and free of use and most are open
source.

\section{Getting Started}
To get started check out the main web site
\code{http://www.openengine.dk}. All OpenEngine resources
can be accessed from the main web page. The site is wiki based and
holds the source code, documentation and useful information regarding
how to get started.

\section{Latest Version}
See how to obtain a copy of the OpenEngine source code at
\code{openengine.dk/wiki/Darcs}. Here you will find a guide on
how to checkout the source code through the version control system
darcs (\code{www.darcs.net}).

\subsection{Documentation}
For the official OpenEngine source code documentation see:
\code{http://openengine.dk/doc/}. To keep the
documentation as updated as possible nightly doxygen builds are
published here.

%\subsection{Projects and Extensions}
%- link and quick explanation of projects and extensions


\subsection{Community}
The website is the main source of new information so start looking
here when questions come to mind. The ongoing discussion on
all kinds of relevant OpenEngine work takes place on the
mailing list, see \code{openengine.dk/wiki/MailingList} on how
to join the list. The mailing list is the best place to ask any
OpenEngine related question.
