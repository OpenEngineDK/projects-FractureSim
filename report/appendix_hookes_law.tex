% -*- mode: latex; mode: auto-fill; coding: utf-8; -*-

\chapter{Hooke's Law in Three Dimensions}
\label{appendix:hookes-law-on-matrix-form}
This section provides the general relationship between stress and
strain for elastic material in three dimensions on matrix form. The
general relationship between stress
and strain for a homogeneous material can be expressed by a
$3 \times 3 \times 3 \times 3$
symmetric fourth-order tensor with 81 components, this tensor is
called the \defit{material tensor} and denoted $C$. But because both
the stress and strain tensors are symmetric $3 \times 3$ matrices the
material tensor reduces to a $6 \times 6$ tensor (matrix) with 36
components for general anisotropic material, this matrix is called the
\defit{material matrix} \citebook{page~220}{book:applied_math}. When
considering isotropic materials only 21 of the 36 components differ
because of symmetry \citebook{page~659-661}{book:fem-engineers}.
%
Using vectorization of the stress and strain matrices and Voigt
notation for the material matrix, the relationship looks as follows:

\begin{equation}
\begin{Bmatrix}
\sigma \\
\tau
\end{Bmatrix}
= C 
\begin{Bmatrix}
\varepsilon \\
\gamma
\end{Bmatrix}
\end{equation}

or on matrix form:

\begin{equation}
\begin{Bmatrix}
\sigma_x \\
\sigma_y \\
\sigma_z \\
\tau_{xy} \\
\tau_{xz} \\
\tau_{yz}
\end{Bmatrix}
=
\begin{bmatrix}
C_{11} & \cdots & \cdots & \cdots & \cdots & C_{16} \\
\vdots & \ddots & & & & \vdots \\
\vdots & & \ddots & & & \vdots \\
\vdots & & & \ddots & & \vdots \\
\vdots & & & & \ddots & \vdots \\
C_{61} & \cdots & \cdots & \cdots & \cdots & C_{66}
\end{bmatrix}
\begin{Bmatrix}
\varepsilon_x \\
\varepsilon_y \\
\varepsilon_z \\
\gamma_{xy} \\
\gamma_{xz} \\
\gamma_{yz}
\end{Bmatrix}
\end{equation}

Depending on if we use linear or non-linear elasticity theory, the
material matrix will look different, here we present the material
matrix for linear elasticity. Recall equation
\eqref{eq:hookes-law-in-3d} from section
\vref{sec:stress-strain-relation-in-3d} relating normal stress and
normal strain, repeated here for convenience.

\begin{subequations}
\begin{equation*}
    \varepsilon_x = \frac {1}{E} \left [ \sigma_x - \nu \left (
        \sigma_y + \sigma_z \right ) \right ]
\end{equation*}
\begin{equation*} 
    \varepsilon_y = \frac {1}{E} \left [ \sigma_y - \nu \left (
        \sigma_x + \sigma_z \right ) \right ]  
\end{equation*}
\begin{equation*}
    \varepsilon_z = \frac {1}{E} \left [ \sigma_z - \nu \left (
        \sigma_x + \sigma_y \right ) \right ]  
\end{equation*}
\end{subequations}

Rewriting this on matrix form the normal strain relation looks as follows:

\begin{equation}
\begin{bmatrix}
\varepsilon_x \\ \varepsilon_y \\ \varepsilon_z
\end{bmatrix}
=
\frac{1}{E}
\begin{bmatrix}
  1 & - \nu & - \nu \\
  - \nu & 1 & - \nu \\
  - \nu & - \nu & 1
\end{bmatrix}
\begin{bmatrix}
  \sigma_x \\ \sigma_y \\ \sigma_z
\end{bmatrix}
\end{equation}

For the shearing stress and strain we recall equation
\eqref{eq:3d-shear-ss-relation} from section
\vref{sec:stress-strain-relation-in-3d}:

\begin{equation*}
  \gamma_{xy} = \tau_{xy} / G
  \qquad
  \gamma_{yz} = \tau_{yz} / G
  \qquad
  \gamma_{zx} = \tau_{zx} / G
\end{equation*}

and equation \eqref{eq:EG-relation} relating
$G$ and $E$

\begin{equation*}
G = \frac{E}{2(1+\nu)}
\end{equation*}

and by substituting $G$ the equations becomes:

\begin{equation}
  \gamma_{xy} = \tau_{xy} \frac{2(1+\nu)}{E}
  \qquad
  \gamma_{yz} = \tau_{yz} \frac{2(1+\nu)}{E}
  \qquad
  \gamma_{zx} = \tau_{zx} \frac{2(1+\nu)}{E}
\end{equation}

on matrix form, it looks like this:

\begin{equation}
\begin{bmatrix}
\gamma_x \\ \gamma_y \\ \gamma_z
\end{bmatrix}
=
\frac{1}{E}
\begin{bmatrix}
  2(1+\nu) & 0 & 0 \\
  0 & 2(1+\nu) & 0 \\
  0 & 0 & 2(1+\nu)
\end{bmatrix}
\begin{bmatrix}
  \tau_x \\ \tau_y \\ \tau_z
\end{bmatrix}
\end{equation}

from the two matrices we assemble the inverse material matrix $D$.

\begin{equation}
\label{eq:D-3d}
D = C^{-1} = \frac{1}{E} 
\begin{bmatrix}
  1 & - \nu & - \nu & 0 & 0 & 0 \\
  - \nu & 1 & - \nu & 0 & 0 & 0 \\
  - \nu & - \nu & 1 & 0 & 0 & 0 \\
  0 & 0 & 0 & 2(1+\nu) & 0 & 0 \\
  0 & 0 & 0 & 0 & 2(1+\nu) & 0 \\
  0 & 0 & 0 & 0 & 0 & 2(1+\nu)
\end{bmatrix}
\end{equation}

the material matrix is then:

\begin{equation}
\label{eq:C-3d}
C = \frac{E}{(1+\nu)(1-2\nu)} 
\begin{bmatrix}
  1-\nu & \nu & \nu & 0 & 0 & 0 \\
  \nu & 1-\nu & \nu & 0 & 0 & 0 \\
  \nu & \nu & 1-\nu & 0 & 0 & 0 \\
  0 & 0 & 0 & \frac{1-2\nu}{2} & 0 & 0 \\
  0 & 0 & 0 & 0 & \frac{1-2\nu}{2} & 0 \\
  0 & 0 & 0 & 0 & 0 & \frac{1-2\nu}{2}
\end{bmatrix}
\end{equation}

Remark that if we should construct the material matrix for the
non-linear elasticity theory, then $C$ would be a function of strain
\citebook{page~659-662}{book:fem-engineers}
